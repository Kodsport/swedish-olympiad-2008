\problemname{Building Pyramids}

\begin{figure}[h!]
\centering
    \includegraphics[width=0.6\textwidth]{pyramid.png}
\caption{An example pyramid of height 3 with 35 blocks.}
\label{fig:pyramid}
\end{figure}

When initiating a larger project, like building a pyramid, it's best to think twice.
Your task today is to write a program that computes how high a pyramid can be built given a certain number of blocks of stone.

We assume that the pyramid to be built is compact, i.e. there are no cavities inside.
Furthermore, we assume it is built according to the principle in Figure~\ref{fig:pyramid}.
Each layer is square, with a side length that is two less than the one below it.
The top layer always consist of a single block.

It is fine if you have leftover blocks, as long as you build a complete pyramid.

\section*{Input}
The first and only line of input contains an integer $N$ ($1 \le N \le 100\,000\,000$), the number of blocks you have available.

\section*{Output}
Output a single integer -- the maximum height of a pyramid that can be built with at most $N$ blocks.

\section*{Scoring}
Your solution will be tested on a set of test groups, each worth a number of points.
To get the points for a test group you need to solve all test cases in the test group. Your final score will be the maximum score of a single submission.

\noindent
\begin{tabular}{| l | l | l |}
  \hline
  Group & Points & Constraints \\ \hline
  $1$    & $33$        &  There will no leftover blocks after building the pyramid. \\ \hline
  $2$    & $67$        &  No additional constraints \\ \hline
\end{tabular}
