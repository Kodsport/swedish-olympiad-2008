\problemname{Röstköp}

Karl-Gunnar kandiderar till ordförandeposten i sin idrottsförening och vill inte riskera att förlora omröstningen.
Han har lyckats få reda på vilken kandidat varje medlem tänker rösta på och tänker helt enkelt muta ett antal
medlemmar så att de röstar på honom istället. Skriv ett program som beräknar hur många röster som måste köpas
(d.v.s. medlemmar som behöver mutas) för att Karl-Gunnar ska vinna omröstningen. För att vinna krävs att man
får fler röster än var och en av de andra kandidaterna.

\section*{Indata}
På första raden står heltalet $n$ ($1 \le n \le 20$), antalet kandidater.

På andra raden står $n$ heltal (mellan $0$ och $1000$): antalet röster varje kandidat skulle få utan mutor.
Det första talet anger Karl-Gunnars röster.

\section*{Utdata}

Skriv ut ett heltal: det minsta antalet röster som behöver köpas för att Karl-Gunnar ska få fler
röster än var och en av de övriga kandidaterna.

\section*{Poängsättning}
Din lösning kommer att testas på en mängd testfallsgrupper.
För att få poäng för en grupp så måste du klara alla testfall i gruppen.

\noindent
\begin{tabular}{| l | l | p{12cm} |}
  \hline
  \textbf{Grupp} & \textbf{Poäng} & \textbf{Gränser} \\ \hline
  $1$    & $40$       & $n \leq 10$ och alla kandidater har som mest 10 röster. \\ \hline
  $2$    & $60$       & Inga ytterligare begränsningar. \\ \hline
\end{tabular}
